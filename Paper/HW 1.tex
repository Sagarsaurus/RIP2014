\documentclass[]{article}
\usepackage{times}
\usepackage{amsmath}
\usepackage{graphicx}
\usepackage{ tipa }
\usepackage{amssymb}
\usepackage{url,hyperref}
\usepackage{enumerate}
%opening
\title{RIP HW 1}
\author{Inder Dhir, Ajmal Kunnummal, Sagar Laud, James Liu, Richard Stauffer}

\begin{document}

\maketitle


\section{Problem 1}
	\begin{enumerate}[(a)]
	\item \textbf{Compare the Methods of the Two Planners:}\\
	Blackbox Planner\\
		This planner operated by turning the given problem into a set of boolean satisfiability problems through two mechanisms. The front end is a graphplan technique where instead of state nodes and edges representing possible traversal, we have nodes of actions/facts and edges from facts -> actions or from actions -> facts affected by said action. Arranged in an alternating fashion: Facts, to possible actions, back to facts. It also uses backward chaining and iterative depth probing to keep from exploring too many extraneous nodes. \\
	
	VHPOP Planner\\
		VHPOP is a partial-order planner. This means it generates a lists of actions necessary to get to the goal, only constraining their order when absolutely necessary (One has preconditions, an earlier action must synthesize) It operates on a system of clever heuristics, and evaluating dead end states.
	
	\item \textbf{Which Planner was Fastest:}\\
	Both took very little time, as this is a relatively small problem in terms of search space, but  the Blackbox planner was still undoubtedly faster at 12 milliseconds versus the 5160 milliseconds of the VHPOP planner. 
	
	\item \textbf{Why Might this Planner be Faster?}\\
	
	\end{enumerate}
\end{document}
